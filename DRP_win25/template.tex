\usepackage{latexsym}
\usepackage{amsmath}
\usepackage{amssymb}
\usepackage{amsthm}
\usepackage{epsfig}
\usepackage{algorithm}
\usepackage{algorithmicx}
\usepackage{algpseudocode}
\usepackage{mathtools}
\usepackage{enumitem}
\usepackage{listings}
\usepackage{cancel}
\usepackage{bbding}
\lstset{
  basicstyle=\ttfamily,
  mathescape
}
%\usepackage{psfig}
\usepackage{comment}
\usepackage{color}
\usepackage{tikz, tikz-cd}
\usepackage{mathrsfs} 
\usetikzlibrary{intersections,positioning}
\usepackage{physics}
\usetikzlibrary{positioning, arrows}
\usetikzlibrary{decorations.pathmorphing, patterns, decorations, decorations.pathreplacing}
\usetikzlibrary{calc}

\newcommand{\handout}[6]{
  \noindent
  \begin{center}
  \framebox{
    \vbox{
      \hbox to 5.78in { {\bf } \hfill #2 }
      \vspace{4mm}
      \hbox to 5.78in { {\Large \hfill #6  \hfill} }
      \vspace{2mm}
      \hbox to 5.78in { {\em #3 \hfill #4} }
      %\vspace{2mm}
      %\hbox to 5.78in { {\em \hfill #5} }
    }
  }
  \end{center}
  \vspace*{4mm}
}

\newcommand{\set}[1]{\left\{#1 \right\}}

\newcommand{\wrap}[1]{\left( #1 \right )}
\newcommand{\bracket}[1]{\left[ #1 \right ]}


\newcommand{\lecture}[5]{\handout{#1}{#2}{#3}{#4}{#5}{#1}}
\theoremstyle{plain}
\newtheorem{theorem}{Theorem}
\newtheorem*{theorem*}{Theorem}
\newtheorem{corollary}[theorem]{Corollary}
\newtheorem*{lemma*}{Lemma}
\newtheorem{lemma}[theorem]{Lemma}
\newtheorem{proposition}[theorem]{Proposition}
\newtheorem{property}{Property}
\newtheorem{claim}[theorem]{Claim}

\theoremstyle{definition}
\newtheorem{definition}{Definition}

\theoremstyle{remark}
\newtheorem{example}{Example}
\newtheorem*{example*}{Example}
\newtheorem*{exercise*}{Exercise}
\newtheorem{fact}[theorem]{Fact}
\newtheorem{assumption}[theorem]{Assumption}
\newtheorem{note}[theorem]{Note}
\newtheorem{observation}{Observation}



% 1-inch margins, from fullpage.sty by H.Partl, Version 2, Dec. 15, 1988.
\topmargin 0pt
\advance \topmargin by -\headheight
\advance \topmargin by -\headsep
\textheight 8.9in
\oddsidemargin 0pt
\evensidemargin \oddsidemargin
\marginparwidth 0.5in
\textwidth 6.5in

\parindent 0in
\parskip 1.5ex
\usepackage{enumitem}
\setlist{itemsep=0pt}
%\renewcommand{\baselinestretch}{1.25}

\newcommand{\OPT}{\mathrm{OPT}}
\DeclareMathOperator*{\argmax}{arg\,max}
\DeclareMathOperator*{\argmin}{arg\,min}

\newcommand{\depth}{\mathrm{depth}}
\newcommand{\dist}{\mathrm{dist}}

\newcommand{\stpath}{s\text{-}t}
\newcommand{\ow}{\textrm{o.w.}}

\newcommand{\LP}{\textrm{LP}}
\newcommand{\ILP}{\textrm{ILP}}
\newcommand{\ex}{\textbf{ Ex. }}

\usepackage{mdframed}
\usepackage{comment}
%\usepackage[noend]{algpseudocode}
\usepackage{algpseudocode}
%\usepackage{varwidth}
%\newcommand{\MyComment}[1]{}

%\usepackage{clrscode3e}
\newcommand{\id}[1]{\mathit{#1}}
\newcommand{\ignore}[1]{}
\newtheorem{invariant}{Invariant}
%\newtheorem*{observation}{Observation}
\newtheorem{remark}{Remark}


\usepackage{tabularx}
\usepackage{subfig}
\usepackage{tikz}
\usepackage{tkz-graph}
\usepackage{ifthen}
\usetikzlibrary{calc}

\usepackage{comment}
\usepackage{enumitem}
\setlist[itemize]{itemsep=0pt}
\usepackage{array}

%%%%%%%%%%%%%%%%%%%%%%%%
% Grading Rubrics
%%%%%%%%%%%%%%%%%%%%%%%%
\usepackage{etoolbox}

\newtoggle{grader}
%\toggletrue{grader}

\mdfdefinestyle{rubric}{frametitle=Grading Rubrics,%
frametitlebackgroundcolor=red!60!black,%
frametitlefont=\normalsize\bfseries\fontfamily{phv}\selectfont\color{white},%
roundcorner=10pt,%
backgroundcolor=red!10,
linecolor=red!60!black,
linewidth=2pt,
}


\newenvironment{rubrics}
{%
\begin{mdframed}[style=rubric]
\fontfamily{phv}\selectfont
\begin{itemize}
}
{%
\end{itemize}
\end{mdframed}
}


\def\dist{\operatorname{dist}}
\def\OPT{\operatorname{OPT}}
\def\OR\mathrm{OR}
\def\AND\mathrm{AND}

\newcommand{\OPTOUT}{\operatorname{OPT}_{\text{out}}}
\newcommand{\OPTIN}{\operatorname{OPT}_{\text{in-or-out}}}

            
% SPACING    
\newcommand{\vcm}[1][1]{\vspace*{#1 cm}}
\newcommand{\hcm}[1][1]{\hspace*{#1 cm}}
\newcommand{\rb}[2]{\raisebox{#1 mm}[0mm][0mm]{#2}}
\newcommand{\istrut}[2][0]{\rule[- #1 mm]{0mm}{#1 mm}\rule{0mm}{#2 mm}}
\newcommand{\zeromath}[1]{\makebox[0mm][l]{$#1$}}
\newcommand{\zero}[1]{\makebox[0mm][r]{#1}}

% MATH
\newcommand{\PP}{{\mathbb P\/}}
\newcommand{\A}{{\mathbb A\/}}
\newcommand{\N}{{\mathbb N\/}}
\newcommand{\Z}{{\mathbb Z\/}}
\newcommand{\C}{{\mathbb C\/}}
\newcommand{\Q}{{\mathbb Q\/}}
\newcommand{\R}{{\mathbb R\/}}
\newcommand{\E}{{\mathbb E\/}}
\newcommand{\V}{{\mathbb V\/}}
\newcommand{\K}{{\mathbb K\/}}
\newcommand{\un}[1]{\underline{#1}}
\newcommand{\paren}[1]{{\left( #1 \right)}}
\newcommand{\bigparen}[1]{\big( #1 \big)}
\newcommand{\biggparen}[1]{\bigg( #1 \bigg)}
\newcommand{\angbrack}[1]{\left< #1 \right>}
\newcommand{\Angbrack}[1]{\Big< #1 \Big>}
\newcommand{\curlybrack}[1]{\left\{ #1 \right\}}
\newcommand{\bigcurly}[1]{\big\{ #1 \big\}}
\newcommand{\biggcurly}[1]{\bigg\{ #1 \big\}}
\newcommand{\sqbrack}[1]{\left[ #1 \right]}
\newcommand{\SqBrack}[1]{\Big[ #1 \Big]}
\newcommand{\SQBrack}[1]{\bigg[ #1 \bigg]}
\newcommand{\Paren}[1]{\Big( #1 \Big)}
\newcommand{\PAREN}[1]{\bigg( #1 \bigg)}
\newcommand{\ceil}[1]{\left\lceil #1 \right\rceil}
\newcommand{\floor}[1]{\left\lfloor #1 \right\rfloor}
\newcommand{\f}[2]{\frac{#1}{#2}}
\newcommand{\fr}[2]{\mbox{$\frac{#1}{#2}$}}
\newcommand{\modulo}{ {\mbox{ \rm mod } }}
\newcommand{\logstar}{{\log^*}}
\newcommand{\bydef}{\stackrel{\operatorname{def}}{=}}
\newcommand{\poly}{\operatorname{poly}}
\newcommand{\polylog}{\operatorname{polylog}}
\newcommand{\bottom}{\perp}
\newcommand{\hookuparrow}{\mathrel{\rotatebox[origin=c]{90}{$\hookrightarrow$}}}
\newcommand{\hookdownarrow}{\mathrel{\rotatebox[origin=c]{-90}{$\hookrightarrow$}}}