%----------------------------------------------------------------------------------------
%    PACKAGES AND THEMES
%----------------------------------------------------------------------------------------

\documentclass[aspectratio=169,xcolor=dvipsnames]{beamer}
\usetheme{SimpleDarkBlue}

\usepackage{hyperref}
\usepackage{graphicx} % Allows including images
\usepackage{booktabs} % Allows the use of \toprule, \midrule and \bottomrule in tables

\usepackage{amsmath}
%----------------------------------------------------------------------------------------
%    TITLE PAGE
%----------------------------------------------------------------------------------------

\title{Integer Points in Dynamical Systems}
\subtitle{PM reading 1 - Dynamics and Arithmetic of Post-critically Finite Polynomials}

\author{Bowen Dai}

\institute
{
    Mentor: Xiao Zhong
}
\date{Auguest 8, 2025} % Date, can be changed to a custom date

%----------------------------------------------------------------------------------------
%    PRESENTATION SLIDES
%----------------------------------------------------------------------------------------

\begin{document}

\begin{frame}
    % Print the title page as the first slide
    \titlepage
\end{frame}


\begin{frame}{Dynamical system}
    Definition:\\
    Let $\phi_1, \dots,\phi_r:\mathbb{P}^1\to \mathbb{P}^1$ be a collections of separable rational maps ($\mathbb{P}^1 = \mathbb{C}\cup\{\infty\}$). The dynamical system generated by $\phi_1,\cdots,\phi_r$ denoted by 
    $$\Phi = \langle\phi_1,\dots,\phi_r\rangle$$
    is defined as all possible compositions of $\phi_i$ where $1\leq i \leq r$, more formally:
    $$\Phi = \{\phi_{i_1}\circ \phi_{i_2}\circ\dots\circ\phi_{i_k}:k\leq0, 1\leq i_j\leq r\}$$\pause\\~
    
    The degree of $\Phi$ is the minimum of degree of all $\phi\in \Phi$\pause\\~
    
    The orbit of a point $P$ is the collection of all points such that $P$ can be mapped into by some $\phi\in\Phi$, denoted as 
    $$O_\Phi^+(P) = \{\phi(P):\phi\in\Phi\}$$
\end{frame}


\begin{frame}{Problem}
    Given a dynamical system, does the orbit of a point contain finitely many integer?
\end{frame}


\begin{frame}{Examples of dynamical system}
    \begin{examples}
        Let $\phi:\mathbb{P}^1\to \mathbb{P}^1$ given by $\phi:z\to z+1$. Then $\Phi_1 = \langle\phi\rangle = \{z+1, z+2, z+3, \dots\}$.    
    \end{examples}\pause
    ~
    
    The question on the number of integer points is easy in this case: if the initial input is an integer, then $\Phi$ can generate infinitely many integer points.\pause \\~
    
    Note that this is true for all $\Phi$ with at least one polynomial with integer coefficient among its generators.
\end{frame}

\begin{frame}{}
    \begin{examples}
        Let $\phi:\mathbb{P}^1\to \mathbb{P}^1$ given by $\phi:z\to \frac{1}{z + 1}$. Then 
        $$\Phi_2 = \langle\phi\rangle = \{\frac{1}{z+1}, \frac{1}{\frac{1}{z+1}+1} = \frac{z+1}{z+2}, \frac{1}{\frac{1}{\frac{1}{z+1}+1}+1} = \frac{z+2}{z+3},\frac{1}{\frac{1}{\frac{1}{\frac{1}{z+1}+1}+1}+1} = \frac{z+3}{z+5} \dots\}$$ 
    \end{examples}\pause 
    ~
    
    In this case, there is no obvious answer if $\Phi$ will generate infinitely many integer points.\pause \\~
    
    However, if we change the coordinate with this coordinate function $y(z) = \frac{1}{z}$. Then the generator of $\Phi_2$ becomes $y\circ\phi = z+1$.\\
    Therefore, $\Phi_2$ is just $\Phi_1$ under a different coordinate system.
\end{frame}

\begin{frame}{Ramification}
    Ramification captures the local invertibility of a function.\pause \\~
    
    Definition:\\
    Ramification index of a function $\phi$ at a point $P$ is denoted as $e_{P}(\phi)$ and defined as 
    $$e_{P}(\phi) = \text{lowest order of non zero derivative of } \phi \text{ at point } P$$
    or more formally, 
    $$e_{P}(\phi) = ord_{P}(\phi(z)-\phi(P))$$ \pause
    
    A function $\phi$ is ramified at a point $P$ if $e_{P}(\phi)>1$
    \pause \\~
    
    $\phi$ is totally ramified at a point $P$ if $e_{P}(\phi) = deg(\phi)$.
\end{frame}

\begin{frame}{Example of ramification index}%present as fact and show example of f(x)=x^d
    \begin{examples}
        Let $\phi(z) = z^3$, then $e_0(\phi)= 3$ and $\phi$ is totally ramified at 0 because the only nonzero derivative of $\phi$ is the third order derivative.
    \end{examples}
    \pause
    
    \begin{examples}
        Let $\phi(z) = (z-1)(z-2)^2$, then $e_1(\phi)=1$ and $e_2(\phi)=2$
    \end{examples}
    
\end{frame}

\begin{frame}{Polynomial}
    Definition: \\
    Let $\phi:\mathbb{P}^1\to\mathbb{P}^1$, be a separable rational map. $\phi$ is a polynomial if any of the following equivalent condition are true:
    \begin{enumerate}
        \item There exist a point $P\in \mathbb{P}^1$ such that $\phi(P) = P$ and $\phi$ is totally ramified at $P$
        \item There exists a coordinate function $y$ (i.e. $y$ is a rational function of degree 1) on $\mathbb{P}^1$ such that $y \circ \phi \in k[y]$
    \end{enumerate}
    If $\phi$ satisfies these conditions, then we say $\phi$ is polynomial with respect to $P$ or that $\phi$ is polynomial with respect to $y$.
\end{frame}

\begin{frame}{Decomposition property of dynamical system}
    Let $\Phi$ be a dynamical system, $P_0\in \mathbb{P}^1$ be a point. We say $\Phi$ has the decomposition property if there exist finite subsets $\Lambda_1, \Lambda_2 \subset \Phi$ satisfying the following conditions:
    \begin{enumerate}
        \item $\displaystyle \Phi = \Lambda_1 \cup \bigcup_{\lambda\in\Lambda_2}\lambda\circ \Phi$
        \item every $\lambda \in \Lambda_2$ satisfies $\#\lambda^{-1}(P_0) \geq 3$
    \end{enumerate}
\end{frame}
\begin{frame}{Polynomial generator and decomposition property}
    \begin{block}{Theorem 1}
        Let $\Phi = \langle \phi_1,\cdots,\phi_r\rangle$ be a dynamical system of degree at least two. Then
        \begin{enumerate}
            \item $\Phi$ has the decomposition property for every $P\in \mathbb{P}^1 \iff \Phi$ contains no nontrivial polynomial maps.
            \item Suppose $\Phi$ is generated by a single element, say $\Phi = \langle \phi \rangle$. Fix a point $P_0\in \mathbb{P}^1$. Then $\Phi$ has the decomposition property $\iff \Phi$ does not contain a nontrivial map which is polynomial with respect to $P_0$. 
        \end{enumerate}
    \end{block}
\end{frame}
\begin{frame}
    \begin{block}{Recall: Theorem 1.1 (forward direction)}
         Let $\Phi = \langle \phi_1,\cdots,\phi_r\rangle$ be a dynamical system of degree at least two. Then:\\~

            $\Phi$ has the decomposition property for every $P\in \mathbb{P}^1 \implies \Phi$ contains no nontrivial polynomial maps.
    \end{block}
\end{frame}
\begin{frame}{Proof part 1}
    Suppose that $\Phi$ has the decomposition property and it has degree at least 2. For the sake of contradiction, assume that $\Phi$ contains a nontrivial polynomial map $\phi \in \Phi$, say $\phi$ is polynomial with respect to the point $P_0\in \mathbb{P}^1$.\pause \\~
    
    As for any iteration of $(\phi^n)^{-1}(P_0) = \{P_0\}$, thus $\phi^n \neq \lambda \psi$ for any $\lambda \in \Lambda_2$.\\~
    (Note that $\phi^n$ denotes the $n^{th}$ iteration of $\phi$, not the $n^{th}$ power of $\phi$.) \pause \\~
    
    Hence, the iterations of $\phi^n \in \Lambda_1$. Since $\Lambda_1$ is finite, there are some $n>m\geq 1$ such that $\phi^n = \phi^m$. Therefore, $deg(\phi) = 1$ which contradict the assumption that $\Phi$ has degree at least 2.
\end{frame}
\begin{frame}
    \begin{block}{Recall: Theorem 1.1 (backward direction)}
         Let $\Phi = \langle \phi_1,\cdots,\phi_r\rangle$ be a dynamical system of degree at least two. Then:\\~
     
        $\Phi$ contains no nontrivial polynomial maps $\implies \Phi$ has the decomposition property for every $P\in \mathbb{P}^1$
    \end{block}
\end{frame}
\begin{frame}
    Now suppose that $\Phi = \langle\phi_1,\dots,\phi_r\rangle$ does not satisfy the decomposition property for some $P_0\in \mathbb{P}^1$, so there is no way to choose finite sets $\Lambda_1,\Lambda_2$ satisfying the decomposition property. \pause \\~
    
    For each integer $m$, let $\Phi_m=\{\phi_{i_1}\circ\phi_{i_2}\circ\cdots\circ\phi_{i_m}: 1\leq i_1,\dots,i_m \leq r\} \subset \Phi$.\pause \\
    Clearly each $\Phi_m$ is finite set, and for each $M\geq 1$, 
    $$\Phi = \bigcup_{m=0}^{M-1}\Phi_m \cup \bigcup_{\lambda\in \Phi_M}\lambda\circ \Phi$$
    \pause
    Since $\Phi$ does not satisfies the decomposition property, every $\Phi_m$ contains a map $\psi_m\in \Phi_m$ such that $\#\psi_m^{-1}(P_0)\leq 2$.\pause \\~
    
    Since $\psi_m \in \Phi_m$, $\psi_m$ can be written as $$\psi_m = \phi_{i_1}\circ\phi_{i_2}\circ\cdots\circ\phi_{i_m}$$ and $$\psi^{-1}_m(P_0)=\phi^{-1}_{i_m}\circ\phi^{-1}_{i_{m-1}}\circ\cdots\circ\phi^{-1}_{i_2}\circ\phi^{-1}_{i_1}(P_0)$$
\end{frame}
\begin{frame}
    Since $\#\psi_m^{-1}(P_0)\leq 2$, there is at most one point $P_t$ such that $\#\phi^{-1}_{i_t}(P_t) = 2$ and for all $i\neq t$, $\#\phi_i^{-1}(P_i) = 1$. \pause \\~
    
    Since this is true for all $m\geq1$, take $m=5r+1$ where $r$ is the number of generators of $\Phi$.\pause \\~
    
    There are only $r$ distinct functions, by pigeon hole principle, there must be a function that appears at least 6 times in the expression for $\psi_m$. Therefore, either $\phi$ appears 3 times before $\phi_t$, or $\phi$ appears 3 times after $\phi_t$. \pause \\~
    
    Call this function $\phi$ and let $P_{u}, P_{v}, P_{w}$ be the inputs of $\phi$.
    $$\cdots P_{u-1}\xleftarrow{\phi}P_u\cdots P_{v-1}\xleftarrow{\phi}P_v\cdots P_{w-1}\xleftarrow{\phi}P_w \cdots$$
\end{frame}
\begin{frame}

    By Hurwitz's formula, we know that $$2deg(\phi)-2\geq\sum_{P\in \mathbb{P}^1}(e_P(\phi)-1)$$ 
    This means that any map can only have at most 2 distinct totally ramified points.\pause \\~
    
    Thus, $P_{u}, P_{v}, P_{w}$ are not all distinct, say $P_{u}=P_{w}$.\pause \\~
    
    Then take the composition of map between $\phi$ at $P_{u}$ and $\phi$ at $P_{w}$. We get a map that has a fixed and totally ramified point which is equivalent as a polynomial.
    $$P_0\xleftarrow{\phi_{i_0}}P_1\xleftarrow{\phi_{i_1}}\dots \xleftarrow{\phi_{i_{u-1}}}P_{u}\underbrace{\xleftarrow{\phi_{i_u} = \phi} P_{u+1}\xleftarrow{\phi_{i_{u+1}}} \dots \xleftarrow{\phi_{i_{w-1}}}}_\text{This is a polynomial with respect to $P_{i_u}$}P_{w} = P_{u}\xleftarrow{\phi_{w}=\phi}\dots$$
\end{frame}

\begin{frame}
    \begin{block}{Recall: Theorem 1.2}
        Let $\Phi = \langle \phi_1,\cdots,\phi_r\rangle$ be a dynamical system of degree at least two. Then:\\~

        Suppose $\Phi$ is generated by a single element, say $\Phi = \langle \phi \rangle$. Fix a point $P_0\in \mathbb{P}^1$. Then $\Phi$ has the decomposition property $\iff \Phi$ does not contain a nontrivial map which is polynomial with respect to $P_0$.
    \end{block}    
\end{frame}

\begin{frame}{Proof part 2}
    Define $\mu, v$ as:
    $$ P_0\underbrace{\xleftarrow{\phi_{i_0}}P_1\xleftarrow{\phi_{i_1}}\dots \xleftarrow{\phi_{i_{u-1}}}}_v P_{u}\underbrace{\xleftarrow{\phi_{i_u}} P_{u+1}\xleftarrow{\phi_{i_{u+1}}} \dots \xleftarrow{\phi_{i_{w-1}}}}_\mu P_{w} = P_{u}\xleftarrow{\phi_{w}}\dots P_m$$ \pause
    
    Since $\Phi$ only has 1 generator $\phi$, $\mu = \phi^n$ and $v=\phi^m$ for some $n,m$.\\
    Note that $v(P_u)=P_0$. \pause \\~
    
    Choose some integer $k$ such that $nk>m$, then consider:
    $$P_u\xrightarrow{\phi^m}P_0\xrightarrow{\phi^{nk-m}}P_u\xrightarrow{\phi^m}P_0\xrightarrow{\phi^{nk-m}}P_u$$
    It follows that $\phi^{nk}$ fixes $P_0$ and $\phi^{nk}$ is totally ramified at $P_0$. Since $\phi^{nk}\in \Phi$, $\Phi$ contains a polynomial with respect to $P_0$. $_\square$
\end{frame}
\begin{frame}{Integer points in dynamical system}
    \begin{block}{Theorem 2}
        Suppose that $\Phi$ contains no polynomial maps and fix a point $P\in \mathbb{P}^1(\mathbb{Q})$. Let $z$ be a coordinate function. Then $$\{Q: Q \in O^+_\Phi(P) \text{ and } z(Q)\in \mathbb{Z} \}$$ is a finite set.
    \end{block}
\end{frame}
\begin{frame}{Proof}
    Since $\Phi$ does not contain any polynomial map, there are finite sets $\Lambda_1, \Lambda_2 \subset \Phi$ such that 
    $$\Phi = \Lambda_1 \cup \bigcup_{\lambda\in\Lambda_2} \lambda\circ\Phi$$ \pause
    Therefore, the orbit of any point P can be written as
    $$O_\Phi(P) = \{\lambda (P): \lambda\in\Lambda_1\} \cup \bigcup_{\lambda \in \Lambda_2}\{\lambda\circ\phi (P): \phi \in \Phi\}$$ \pause
    Thus, it is equivalent to show the set $\{\lambda\circ\phi (P): \phi \in \Phi\text{ and } z\circ\lambda\circ\phi (P) \in \mathbb{Z} \}$ is finite for each $\lambda \in \Lambda_2$.
    
\end{frame}

\begin{frame}
    For any homogeneous coordinate $[X,Y]$ on $\mathbb{P}^1$, let:\\
    \begin{itemize}
        \item $z$ be the coordinate function with $z = \frac{X}{Y}$
        \item for any $\phi\in \Phi$, write $\phi P = [u_\phi,v_\phi]$
        \item $\lambda = [F_\lambda, G_\lambda]$ where $F_\lambda, G_\lambda \in \mathbb{Z}[X,Y]$ are homogeneous polynomial of degree $d$
    \end{itemize}
    
    Since $F_\lambda$, $G_\lambda$ have no common factors. Thus, the resultant of $F_\lambda$ and $G_\lambda$, $Res(F_\lambda,G_\lambda)\neq 0$.
\end{frame}

\begin{frame}
    We observe that
    $$z\circ\lambda\circ\phi (P)\in \mathbb{Z} \iff \frac{F_\lambda(u_\phi, v_\phi)}{G_\lambda(u_\phi, v_\phi)}\in \mathbb{Z} \pause \implies \frac{Res(F_\lambda, G_\lambda)}{G_\lambda(u_\phi, v_\phi)}\in \mathbb{Z}$$ \pause
    Since $Res(F_\lambda, G_\lambda)$ is fixed and independent of $\phi$, there are finitely possibility for $G_\lambda(u_\phi, v_\phi)$.\pause \\~

    Since $\# \lambda^{-1}(P_0) \geq 3$, $deg(G_\lambda)\geq3$. It is a Thue-Mahler equation, so there are only finitely many co-prime paire of $[u_\phi,v_\phi]$ satisfying the equation.\pause \\~
    
    Thus, for any $\lambda \in \Lambda_2$, the set $\{\lambda\phi P: \phi \in \Phi \text{ and } z(\lambda\phi P)\in \mathbb{Z}\}$ is finite which is equivalent as $\{Q: Q \in O^+_\Phi(P) \text{ and } z(Q)\in \mathbb{Z}\}$ is a finite set. $_\square$
\end{frame}

\begin{frame}{Integer points in dynamical system generated by a single function}
    \begin{block}{Theorem 3}
        Let $\phi(Z)\in K(Z)$ be a rational function of degree at least two and let $t \in K \cup \{\infty\} = \mathbb{P}^1(K)$. If $\phi^2(Z)\not\in \bar{K}[Z]$, then the sequence 
        $$t, \phi(t),\phi^2(t), \phi^3(t),\dots$$
        contains only finitely many elements of $R_S$.
    \end{block}
\end{frame}

\begin{frame}
    \Huge{\centerline{\textbf{Thank you!}}}
\end{frame}

\begin{frame}{Reference}
    J.~H. Silverman, Integer points, Diophantine approximation, and iteration of rational maps, Duke Math. J. {\bf 71} (1993), no.~3, 793--829
\end{frame}

\end{document}